%%%%%%%%%%%%%%%%%%%%%%%%%%%%%%%%%%%%%%%%%
% Jacobs Landscape Poster
% LaTeX Template
% Version 1.1 (14/06/14)
%
% Created by:
% Computational Physics and Biophysics Group, Jacobs University
% https://teamwork.jacobs-university.de:8443/confluence/display/CoPandBiG/LaTeX+Poster
% 
% Further modified by:
% Nathaniel Johnston (nathaniel@njohnston.ca)
%
% This template has been downloaded from:
% http://www.LaTeXTemplates.com
%
% License:
% CC BY-NC-SA 3.0 (http://creativecommons.org/licenses/by-nc-sa/3.0/)
%
%%%%%%%%%%%%%%%%%%%%%%%%%%%%%%%%%%%%%%%%%

%----------------------------------------------------------------------------------------
%	PACKAGES AND OTHER DOCUMENT CONFIGURATIONS
%----------------------------------------------------------------------------------------

\documentclass[final]{beamer}

\usepackage[scale=1.0]{beamerposter} % Use the beamerposter package for laying out the poster
\usepackage[acronym,toc]{glossaries}
\input{../acros}
\usetheme{confposter} % Use the confposter theme supplied with this template

\setbeamercolor{block title}{fg=dblue!80,bg=white} % Colors of the block titles
\setbeamercolor{block body}{fg=black,bg=white} % Colors of the body of blocks
\setbeamercolor{block alerted title}{fg=white,bg=dblue!70} % Colors of the highlighted block titles
\setbeamercolor{block alerted body}{fg=black,bg=dblue!10} % Colors of the body of highlighted blocks
% Many more colors are available for use in beamerthemeconfposter.sty

%-----------------------------------------------------------
% Define the column widths and overall poster size
% To set effective sepwid, onecolwid and twocolwid values, first choose how many columns you want and how much separation you want between columns
% In this template, the separation width chosen is 0.024 of the paper width and a 4-column layout
% onecolwid should therefore be (1-(# of columns+1)*sepwid)/# of columns e.g. (1-(4+1)*0.024)/4 = 0.22
% onecolwid should therefore be (1-(# of columns+1)*sepwid)/# of columns e.g. 
% (1-(3+1)*0.025)/3 = 0.3
% Set twocolwid to be (2*onecolwid)+sepwid = 0.464
% Set threecolwid to be (3*onecolwid)+2*sepwid = 0.708

\newlength{\sepwid}
\newlength{\onecolwid}
\newlength{\twocolwid}
\newlength{\threecolwid}
\setlength{\paperwidth}{36in} % A0 width: 46.8in
\setlength{\paperheight}{48in} % A0 height: 33.1in
\setlength{\textwidth}{34in} % A0 width: 46.8in
\setlength{\textheight}{46in} % A0 height: 33.1in
\setlength{\sepwid}{0.025\paperwidth} % Separation width (white space) between columns
\setlength{\onecolwid}{0.3\paperwidth} % Width of one column
\setlength{\twocolwid}{0.625\paperwidth} % Width of two columns
\setlength{\threecolwid}{0.95\paperwidth} % Width of three columns
\setlength{\topmargin}{-0.5in} % Reduce the top margin size
%-----------------------------------------------------------

\usepackage{graphicx}  % Required for including images
\newcommand{\Cyclus}{\textsc{Cyclus}\xspace}%
\usepackage{tabularx}
\newcolumntype{b}{X}
\newcolumntype{s}{>{\hsize=.5\hsize}X}
\newcolumntype{m}{>{\hsize=.75\hsize}X}
\newcolumntype{z}{>{\hsize=.65\hsize}X}

\usepackage{booktabs} % Top and bottom rules for tables
\usepackage{xspace}
\usepackage{amsmath}
\usepackage{exscale}

\setbeamertemplate{bibliography item}[text]

%----------------------------------------------------------------------------------------
%	TITLE SECTION 
%----------------------------------------------------------------------------------------

\title{%
  \texorpdfstring{%
    \makebox[\linewidth]{%
      \makebox[0pt][l]{%
        \raisebox{\dimexpr-\height+\baselineskip}[0pt][0pt]
          {\includegraphics[height=2.5\baselineskip]{UIUC_Logo}}% Left logo
      }\hfill
      \makebox[0pt]{Title}%
      \hfill\makebox[0pt][r]{%
        \raisebox{\dimexpr-\height+\baselineskip}[0pt][0pt]
          {\includegraphics[height=3.3\baselineskip]{arfc_atom}}% Right logo
      }%
    }%
  }
  % To make a title with two lines, remove the "%" from the following two lines 
  %{First line of title}
  %{Second line of title}
  {\vspace{1cm}}
  } % Poster title

\author{\textbf{Primary Author}, Other Authors}
\institute{University of Illinios at Urbana-Champaign, Department of Nuclear, Plasma, and Radiological Engineering, Urbana, IL 61801}
%----------------------------------------------------------------------------------------

\begin{document}

\addtobeamertemplate{block end}{}{\vspace*{2ex}} % White space under blocks
\addtobeamertemplate{block alerted end}{}{\vspace*{2ex}} % White space under highlighted (alert) blocks

\setlength{\belowcaptionskip}{2ex} % White space under figures
\setlength\belowdisplayshortskip{2ex} % White space under equations

\begin{frame}[t] % The whole poster is enclosed in one beamer frame

\begin{columns}[t,totalwidth=\threecolwid] % The whole poster consists of three major columns, the second of which is split into two columns twice - the [t] option aligns each column's content to the top

\begin{column}{0.5\sepwid}\end{column} % Empty spacer column

\begin{column}{\onecolwid} % The first column

%----------------------------------------------------------------------------------------
%	INTRODUCTION
%----------------------------------------------------------------------------------------

\begin{block}{Introduction}

\textbf{Sub-title}

Example un-ordered list: 
\begin{itemize}
	\item Example acronym call \gls{SNF} with an example citation 
	    \cite{call_tag_article}
	\item second bullet point with second use of an acronym \gls{SNF}
\end{itemize}

\vspace{0.7em}
\textbf{Sub-title}

More text relating to this section, and a third use of the same acronym
 \gls{SNF}. 

These goals will be achieved by: 
\begin{itemize}
	\item Step to goal 
	\item Step to goal 
\end{itemize}

\end{block}

%----------------------------------------------------------------------------------------
%	OBJECTIVES
%----------------------------------------------------------------------------------------

%This section creates an orange border around a white box
\setbeamercolor{block alerted title}{fg=black,bg=norange} % Change the alert block title colors
\setbeamercolor{block alerted body}{fg=black,bg=white} % Change the alert block body colors
\begin{alertblock}{Objectives}
\begin{itemize}
        \item Objective 1
	\item Objective 2
\end{itemize}

\end{alertblock}

%----------------------------------------------------------------------------------------
%	Programs
%----------------------------------------------------------------------------------------

\begin{block}{Program 1}
Description

\begin{figure}
	\label{fig:figure_label_ex1}
	\includegraphics[width=0.9\linewidth]{graphic_name1}
	\caption{Caption1 Reference \cite{call_tag_article}}
\end{figure}

\end{block}

\begin{block}{Program 2}

Description
 
\begin{itemize}
	\item{Item 1}
	\item{Item 2}
	\item{Item 3}
\end{itemize}

\end{block}

%----------------------------------------------------------------------------------------

\end{column} % End of the first column

\begin{column}{\sepwid}\end{column} % Empty spacer column


%----------------------------------------------------------------------------------------

\begin{column}{\onecolwid} % The second column
%----------------------------------------------------------------------------------------
%	MODELS
%----------------------------------------------------------------------------------------

\begin{block}{Models}
\vspace{0.7em}
\textbf{First Model}

Description

Example table 
\begin{table}[]
	\label{tab:table_label}
	\caption{Table caption with citation \cite{call_tag_article}}
	\begin{tabular}{|l|l|l|l|}
	\hline
	Header1 &  Header2 [Units]  &  Header3 [Units]  &  Header3 [Units]   \\ \hline
	Item 1     & 100 & 2.5  & 2.5\\ \hline
	Iten 2      & 100 & 2.5 & 2.5\\ \hline
	Item 3     & 100 & 2.5  & 2.5\\ \hline
	\end{tabular}
\end{table}

In the waste repository model, the user can define the variables: 
	\begin{itemize}
		\item Variable
		\item Variable
		\item Variable 
		\item Variable 
		\item Variable
	\end{itemize}

\vspace{0.7em}
\textbf{Second Model}

Description 

\vspace{0.7em}
\textbf{Third Model}

Description with an example of a figure reference

Figure \ref{fig:figure_label_ex1} does stuff 

\begin{figure}
	\label{fig:figure_label_ex2}
	\includegraphics[width=1\linewidth]{graphic_name2}
	\caption{Caption2 Reference \cite{call_tag_article}}
\end{figure}

Description 

Equations
\begin{align*}
	y = x
\end{align*}
Description
\begin{align*}
	y = x
\end{align*}
Description
\begin{align*}
	y = x
\end{align*}

\end{block}


%----------------------------------------------------------------------------------------

\end{column} % End of column 2

\begin{column}{\sepwid}\end{column} % Empty spacer column

\begin{column}{\onecolwid} % The third column

\begin{block}{Results}
\textbf{Something}

Description/analysis
 
Contextualization
Figure \ref{fig:figure_label_ex1} illustrates something.

Description

\begin{figure}
	\label{fig:figure_label_ex3}
	\includegraphics[width=0.9\linewidth]{graphic_name3}
	\caption{Caption3 Reference \cite{call_tag_article}}
\end{figure}

\end{block}

% This section creates an box with an orange border and a white background
\setbeamercolor{block alerted title}{fg=black,bg=norange} % Change the alert block title colors
\setbeamercolor{block alerted body}{fg=black,bg=white} % Change the alert block body colors
\begin{alertblock}{Future Work }
\begin{itemize}
		\item Future work
\end{itemize}

\end{alertblock}


%----------------------------------------------------------------------------------------
%	ACKNOWLEDGEMENTS
%----------------------------------------------------------------------------------------

\setbeamercolor{block title}{fg=norange,bg=white} % Change the block title color

\begin{block}{Acknowledgements}
	
	Text specific to your funding, and other acknowledgements you'd like to make
	
\end{block}

%----------------------------------------------------------------------------------------
%	CONTACT INFORMATION
%----------------------------------------------------------------------------------------

\setbeamercolor{block alerted title}{fg=white,bg=dblue} % Change the alert block title colors
\setbeamercolor{block alerted body}{fg=black,bg=white} % Change the alert block body colors

\begin{alertblock}{Contact Information}
	\setbeamercolor{block title}{fg=norange,bg=white} % Change the block title color
	\begin{itemize}
		\item Email: \href{mailto:netid@illinois.edu}{netid@illinois.edu}
		\item Phone: \href{tel:15555555555}{+1 555 555 5555}
	\end{itemize}
% These specific elements are optional, but you should have some method for people to contact you.	
	
\end{alertblock}

\begin{block}{References}

	{\footnotesize\bibliographystyle{abbrv} 
	\bibliography{poster}}
\end{block}


%----------------------------------------------------------------------------------------



\end{column} % End of the third column

\end{columns} % End of all the columns in the poster

\end{frame} % End of the enclosing frame

\end{document}
\begin{column}{\sepwid}\end{column} % Empty spacer column
